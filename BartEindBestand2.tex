\documentclass[]{article}
\usepackage{lmodern}
\usepackage{amssymb,amsmath}
\usepackage{ifxetex,ifluatex}
\usepackage{fixltx2e} % provides \textsubscript
\ifnum 0\ifxetex 1\fi\ifluatex 1\fi=0 % if pdftex
  \usepackage[T1]{fontenc}
  \usepackage[utf8]{inputenc}
\else % if luatex or xelatex
  \ifxetex
    \usepackage{mathspec}
  \else
    \usepackage{fontspec}
  \fi
  \defaultfontfeatures{Ligatures=TeX,Scale=MatchLowercase}
\fi
% use upquote if available, for straight quotes in verbatim environments
\IfFileExists{upquote.sty}{\usepackage{upquote}}{}
% use microtype if available
\IfFileExists{microtype.sty}{%
\usepackage{microtype}
\UseMicrotypeSet[protrusion]{basicmath} % disable protrusion for tt fonts
}{}
\usepackage[margin=1in]{geometry}
\usepackage{hyperref}
\hypersetup{unicode=true,
            pdftitle={Notebook Bart Vermeulen 2019},
            pdfborder={0 0 0},
            breaklinks=true}
\urlstyle{same}  % don't use monospace font for urls
\usepackage{color}
\usepackage{fancyvrb}
\newcommand{\VerbBar}{|}
\newcommand{\VERB}{\Verb[commandchars=\\\{\}]}
\DefineVerbatimEnvironment{Highlighting}{Verbatim}{commandchars=\\\{\}}
% Add ',fontsize=\small' for more characters per line
\usepackage{framed}
\definecolor{shadecolor}{RGB}{248,248,248}
\newenvironment{Shaded}{\begin{snugshade}}{\end{snugshade}}
\newcommand{\AlertTok}[1]{\textcolor[rgb]{0.94,0.16,0.16}{#1}}
\newcommand{\AnnotationTok}[1]{\textcolor[rgb]{0.56,0.35,0.01}{\textbf{\textit{#1}}}}
\newcommand{\AttributeTok}[1]{\textcolor[rgb]{0.77,0.63,0.00}{#1}}
\newcommand{\BaseNTok}[1]{\textcolor[rgb]{0.00,0.00,0.81}{#1}}
\newcommand{\BuiltInTok}[1]{#1}
\newcommand{\CharTok}[1]{\textcolor[rgb]{0.31,0.60,0.02}{#1}}
\newcommand{\CommentTok}[1]{\textcolor[rgb]{0.56,0.35,0.01}{\textit{#1}}}
\newcommand{\CommentVarTok}[1]{\textcolor[rgb]{0.56,0.35,0.01}{\textbf{\textit{#1}}}}
\newcommand{\ConstantTok}[1]{\textcolor[rgb]{0.00,0.00,0.00}{#1}}
\newcommand{\ControlFlowTok}[1]{\textcolor[rgb]{0.13,0.29,0.53}{\textbf{#1}}}
\newcommand{\DataTypeTok}[1]{\textcolor[rgb]{0.13,0.29,0.53}{#1}}
\newcommand{\DecValTok}[1]{\textcolor[rgb]{0.00,0.00,0.81}{#1}}
\newcommand{\DocumentationTok}[1]{\textcolor[rgb]{0.56,0.35,0.01}{\textbf{\textit{#1}}}}
\newcommand{\ErrorTok}[1]{\textcolor[rgb]{0.64,0.00,0.00}{\textbf{#1}}}
\newcommand{\ExtensionTok}[1]{#1}
\newcommand{\FloatTok}[1]{\textcolor[rgb]{0.00,0.00,0.81}{#1}}
\newcommand{\FunctionTok}[1]{\textcolor[rgb]{0.00,0.00,0.00}{#1}}
\newcommand{\ImportTok}[1]{#1}
\newcommand{\InformationTok}[1]{\textcolor[rgb]{0.56,0.35,0.01}{\textbf{\textit{#1}}}}
\newcommand{\KeywordTok}[1]{\textcolor[rgb]{0.13,0.29,0.53}{\textbf{#1}}}
\newcommand{\NormalTok}[1]{#1}
\newcommand{\OperatorTok}[1]{\textcolor[rgb]{0.81,0.36,0.00}{\textbf{#1}}}
\newcommand{\OtherTok}[1]{\textcolor[rgb]{0.56,0.35,0.01}{#1}}
\newcommand{\PreprocessorTok}[1]{\textcolor[rgb]{0.56,0.35,0.01}{\textit{#1}}}
\newcommand{\RegionMarkerTok}[1]{#1}
\newcommand{\SpecialCharTok}[1]{\textcolor[rgb]{0.00,0.00,0.00}{#1}}
\newcommand{\SpecialStringTok}[1]{\textcolor[rgb]{0.31,0.60,0.02}{#1}}
\newcommand{\StringTok}[1]{\textcolor[rgb]{0.31,0.60,0.02}{#1}}
\newcommand{\VariableTok}[1]{\textcolor[rgb]{0.00,0.00,0.00}{#1}}
\newcommand{\VerbatimStringTok}[1]{\textcolor[rgb]{0.31,0.60,0.02}{#1}}
\newcommand{\WarningTok}[1]{\textcolor[rgb]{0.56,0.35,0.01}{\textbf{\textit{#1}}}}
\usepackage{longtable,booktabs}
\usepackage{graphicx,grffile}
\makeatletter
\def\maxwidth{\ifdim\Gin@nat@width>\linewidth\linewidth\else\Gin@nat@width\fi}
\def\maxheight{\ifdim\Gin@nat@height>\textheight\textheight\else\Gin@nat@height\fi}
\makeatother
% Scale images if necessary, so that they will not overflow the page
% margins by default, and it is still possible to overwrite the defaults
% using explicit options in \includegraphics[width, height, ...]{}
\setkeys{Gin}{width=\maxwidth,height=\maxheight,keepaspectratio}
\IfFileExists{parskip.sty}{%
\usepackage{parskip}
}{% else
\setlength{\parindent}{0pt}
\setlength{\parskip}{6pt plus 2pt minus 1pt}
}
\setlength{\emergencystretch}{3em}  % prevent overfull lines
\providecommand{\tightlist}{%
  \setlength{\itemsep}{0pt}\setlength{\parskip}{0pt}}
\setcounter{secnumdepth}{0}
% Redefines (sub)paragraphs to behave more like sections
\ifx\paragraph\undefined\else
\let\oldparagraph\paragraph
\renewcommand{\paragraph}[1]{\oldparagraph{#1}\mbox{}}
\fi
\ifx\subparagraph\undefined\else
\let\oldsubparagraph\subparagraph
\renewcommand{\subparagraph}[1]{\oldsubparagraph{#1}\mbox{}}
\fi

%%% Use protect on footnotes to avoid problems with footnotes in titles
\let\rmarkdownfootnote\footnote%
\def\footnote{\protect\rmarkdownfootnote}

%%% Change title format to be more compact
\usepackage{titling}

% Create subtitle command for use in maketitle
\providecommand{\subtitle}[1]{
  \posttitle{
    \begin{center}\large#1\end{center}
    }
}

\setlength{\droptitle}{-2em}

  \title{Notebook Bart Vermeulen 2019}
    \pretitle{\vspace{\droptitle}\centering\huge}
  \posttitle{\par}
    \author{}
    \preauthor{}\postauthor{}
      \predate{\centering\large\emph}
  \postdate{\par}
    \date{Oktober 2019}


\begin{document}
\maketitle

De gemakkelijkste manier om met SQL aan de slag te gaan in RStudio, is
gebruik te maken van de ingebouwde sqlite database. We kunnen een excel
document inlezen maar deze moeten we dan middels een paar commando's
opslaan in de ingebouwde sqlite database. Die instructies krijg je van
ons.

We starten met een uitgebreide setup chunk waarin we:

\begin{enumerate}
\def\labelenumi{\arabic{enumi}.}
\tightlist
\item
  alle packages laden die we gebruiken;
\item
  de data van een excel file inlezen in een R data frame
\item
  een in-memory sqlite database maken
\item
  het ingelezen R data frame als table in de sqlite database copieren
\end{enumerate}

Benodigde libraries

Lees het excel bestand in

Deze code is nodig om SQL statements los te kunnen laten op het excel
bestand.

\begin{Shaded}
\begin{Highlighting}[]
\CommentTok{#Onderstaande niet veranderen !}
\NormalTok{con <-}\StringTok{ }\NormalTok{DBI}\OperatorTok{::}\KeywordTok{dbConnect}\NormalTok{(RSQLite}\OperatorTok{::}\KeywordTok{SQLite}\NormalTok{(), }\StringTok{":memory:"}\NormalTok{)}


\CommentTok{#Als er in de tabel variabelen met datums voorkomen, dan moeten die aangepast worden zodat de datumvelden alst teksttype worden opgeslagen en niet als datumtijd type. Zie onderstaand voorbeeld: copy_to(con, mutate_at(df, vars(ends_with("datum")), as.character), "factOrders")}
\CommentTok{#copy_to(con, mutate_at(df, vars(ends_with("datum")), as.character), "factOrders")}
\KeywordTok{copy_to}\NormalTok{(con,df, }\StringTok{"MijnTabel"}\NormalTok{)}


\NormalTok{knitr}\OperatorTok{::}\NormalTok{opts_chunk}\OperatorTok{$}\KeywordTok{set}\NormalTok{(}\DataTypeTok{connection =} \StringTok{"con"}\NormalTok{)}
\end{Highlighting}
\end{Shaded}

Welke verschillende producten zijn er verkocht?

\begin{Shaded}
\begin{Highlighting}[]
\KeywordTok{SELECT} \KeywordTok{DISTINCT}\NormalTok{ Product}
\KeywordTok{from}\NormalTok{ MijnTabel}
\KeywordTok{where}\NormalTok{ Jaar }\OperatorTok{=} \DecValTok{2013}
\end{Highlighting}
\end{Shaded}

\begin{longtable}[]{@{}l@{}}
\caption{Displaying records 1 - 10}\tabularnewline
\toprule
Product\tabularnewline
\midrule
\endfirsthead
\toprule
Product\tabularnewline
\midrule
\endhead
Hitch Rack - 4-Bike\tabularnewline
All-Purpose Bike Stand\tabularnewline
Water Bottle - 30 oz.\tabularnewline
Mountain Bottle Cage\tabularnewline
Road Bottle Cage\tabularnewline
AWC Logo Cap\tabularnewline
Bike Wash - Dissolver\tabularnewline
Fender Set - Mountain\tabularnewline
Half-Finger Gloves, L\tabularnewline
Half-Finger Gloves, M\tabularnewline
\bottomrule
\end{longtable}

Welke

\begin{Shaded}
\begin{Highlighting}[]
\KeywordTok{select} \KeywordTok{distinct} \OperatorTok{*}
\KeywordTok{from}\NormalTok{ MijnTabel}
\end{Highlighting}
\end{Shaded}

Vervolgens kunnen we dit dataframe gebruiken als een reguliere R
dataframe. We kunnen het dus gebruiken als input voor ggplot.

\begin{Shaded}
\begin{Highlighting}[]
\KeywordTok{ggplot}\NormalTok{(df_result, }\KeywordTok{aes}\NormalTok{(KlantID, Leeftijd)) }\OperatorTok{+}
\StringTok{  }\KeywordTok{geom_point}\NormalTok{()}
\end{Highlighting}
\end{Shaded}

\includegraphics{BartEindBestand_files/figure-latex/unnamed-chunk-4-1.pdf}

Hoeveel verschillende leeftijden een fiets hebben gekocht

\begin{Shaded}
\begin{Highlighting}[]
\KeywordTok{select} \FunctionTok{COUNT}\NormalTok{(}\KeywordTok{DISTINCT}\NormalTok{ leeftijd)}
\KeywordTok{from}\NormalTok{ MijnTabel}
\end{Highlighting}
\end{Shaded}

\begin{longtable}[]{@{}l@{}}
\caption{1 records}\tabularnewline
\toprule
COUNT(DISTINCT leeftijd)\tabularnewline
\midrule
\endfirsthead
\toprule
COUNT(DISTINCT leeftijd)\tabularnewline
\midrule
\endhead
70\tabularnewline
\bottomrule
\end{longtable}

GGPLOT gedeelte


\end{document}
